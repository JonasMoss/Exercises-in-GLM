\section*{Additional Exercise 1}
It's given that
\begin{align*}
\bm{\mathrm{a}} = 
\begin{bmatrix}
a_1\\
\vdots\\
a_p
\end{bmatrix}
, \quad
\bm{\beta} = 
\begin{bmatrix}
\beta_1\\
\vdots\\
\beta_p
\end{bmatrix}
, \quad
\bm{\mathrm{A}} =
\begin{bmatrix}
a_{1,1} & \cdots & a_{1,p} \\
\vdots & \ddots & \vdots \\
a_{p,1} & \cdots & a_{p,p} \\
\end{bmatrix}
\end{align*}
and
\begin{align*}
\frac{\partial \bm{\mathrm{a}}^{\rm T}\bm{\beta}}{\partial \bm{\beta}}
=
\begin{bmatrix}
\frac{\partial \bm{\mathrm{a}}^{\rm T}\bm{\beta}}{\partial \beta_{1}}\\
\vdots\\
\frac{\partial \bm{\mathrm{a}}^{\rm T}\bm{\beta}}{\partial \beta_{p}}\\
\end{bmatrix}
, \quad
\frac{\partial \bm{\beta}^{\rm T}\bm{\mathrm{A}}\bm{\beta}}{\partial \bm{\beta}}
=
\begin{bmatrix}
\frac{\partial \bm{\beta}^{\rm T}\bm{\mathrm{A}}\bm{\beta}}{\partial \beta_{1}}\\
\vdots\\
\frac{\partial \bm{\beta}^{\rm T}\bm{\mathrm{A}}\bm{\beta}}{\partial \beta_{p}}\\
\end{bmatrix}
.
\end{align*}


\subsection*{(i)}

Since 
\begin{align*}
\bm{\mathrm{a}}^{\rm T}\bm{\beta} =
\sum_{j=1}^{p} a_{j}\beta_{j}
,
\end{align*} we have
\begin{align*}
\frac{\partial \bm{\mathrm{a}}^{\rm T}\bm{\beta}}{\partial \bm{\beta}}
=
\begin{bmatrix}
\frac{\partial \bm{\mathrm{a}}^{\rm T}\bm{\beta}}{\partial \beta_{1}}\\
\vdots\\
\frac{\partial \bm{\mathrm{a}}^{\rm T}\bm{\beta}}{\partial \beta_{p}}\\
\end{bmatrix}
=
\begin{bmatrix}
\frac{\partial \sum_{j=1}^{p} a_{j}\beta_{j}}{\partial \beta_{1}}\\
\vdots\\
\frac{\partial \sum_{j=1}^{p} a_{j}\beta_{j}}{\partial \beta_{p}}\\
\end{bmatrix}
=
\begin{bmatrix}
a_{1}\\
\vdots\\
a_{p}\\
\end{bmatrix}
=
\bm{\mathrm{a}}.
\end{align*}


\subsection*{(ii)}

First, we obtain the element-wise expression for the right side term.
\begin{align*}
(\bm{\mathrm{A}} + \bm{\mathrm{A}}^{\rm T})\bm{\beta}
= \bm{\mathrm{A}}\bm{\beta} + \bm{\mathrm{A}}^{\rm T}\bm{\beta}
= 
\begin{bmatrix}
\sum_{j=1}^{p} a_{1,j}\beta_{j}\\
\vdots\\
\sum_{j=1}^{p} a_{p,j}\beta_{j}\\
\end{bmatrix}
+
\begin{bmatrix}
\sum_{i=1}^{p} a_{i,1}\beta_{i}\\
\vdots\\
\sum_{i=1}^{p} a_{i,p}\beta_{i}\\
\end{bmatrix}.
\end{align*}

Then, we show that the left side term is equal to it. Since 
\begin{align*}
\bm{\beta}^{\rm T} \bm{\mathrm{A}} \bm{\beta} =
\begin{bmatrix}
\beta_{1} \cdots \beta_{p}
\end{bmatrix}
\cdot
\begin{bmatrix}
a_{1,1} & \cdots & a_{1,p} \\
\vdots & \ddots & \vdots \\
a_{p,1} & \cdots & a_{p,p} \\
\end{bmatrix}
\cdot
\begin{bmatrix}
\beta_{1}\\
\vdots\\
\beta_{p}
\end{bmatrix}
= \sum_{i=1}^{p}\beta_{i} \sum_{j=1}^{p} a_{i,j}\beta_{j}
= \sum_{i=1}^{p} \sum_{j=1}^{p} a_{i,j} \beta_{i} \beta_{j},
\end{align*}
we have
\begin{align*}
\frac{\partial \bm{\beta}^{\rm T}\bm{\mathrm{A}}\bm{\beta}}{\partial \bm{\beta}}
&=
\begin{bmatrix}
\frac{\partial \bm{\beta}^{\rm T}\bm{\mathrm{A}}\bm{\beta}}{\partial \beta_{1}}\\
\vdots\\
\frac{\partial \bm{\beta}^{\rm T}\bm{\mathrm{A}}\bm{\beta}}{\partial \beta_{p}}\\
\end{bmatrix}
=
\begin{bmatrix}
\frac{\partial \sum_{i=1}^{p} \sum_{j=1}^{p} a_{i,j} \beta_{i} \beta_{j}}{\partial \beta_{1}}\\
\vdots\\
\frac{\partial \sum_{i=1}^{p} \sum_{j=1}^{p} a_{i,j} \beta_{i} \beta_{j}}{\partial \beta_{p}}\\
\end{bmatrix}
\\
&=
\begin{bmatrix}
\sum_{i=1}^{p} \sum_{j=1}^{p} \frac{ \partial a_{i,j} \beta_{i} \beta_{j}}{\partial \beta_{1}}\\
\vdots\\
\sum_{i=1}^{p} \sum_{j=1}^{p} \frac{\partial a_{i,j} \beta_{i} \beta_{j}}{\partial \beta_{p}}\\
\end{bmatrix}
=
\begin{bmatrix}
\sum_{j=1}^{p} a_{1,j} \beta_{j} + \sum_{i=1}^{p} a_{i,1} \beta_{i}\\
\vdots\\
\sum_{j=1}^{p} a_{p,j} \beta_{j} + \sum_{i=1}^{p} a_{i,p} \beta_{i}\\
\end{bmatrix}
\\
&= (\bm{\mathrm{A}} + \bm{\mathrm{A}}^{\rm T})\bm{\beta}.
\end{align*}

\vspace{1cm}
There are many differentiation rules that will come in handy. You are not likely to take a course covering matrix differentiation, but it is one of the most useful techniques to be acquainted with. An excellent source for formulas in matrix differentiation is \href{https://www.math.uwaterloo.ca/~hwolkowi/matrixcookbook.pdf}{The Matrix Cookbook} by Kaare Brandt Petersen and Michael Syskind Pedersen. The bible of matrix differentiation is the book by Magnus and Neudecker \cite{Magnus2019-cz}. For the theoretical rationale behind matrix differentiation, take a look at Chapter 6 in Tom Lindstrøm's \href{https://www.uio.no/studier/emner/matnat/math/MAT2400/v16/spaces.pdf}{Mathematical Analysis}.

In particular, you should know the following:

\begin{framed}
Let the scalar $\alpha$ be defined by $\alpha = \bm{b}^{\rm T}\bm{A}\bm{x}$~~where $\bm{b}$ and $\bm{A}$ are not a function of $\bf{x}$, then
\begin{align*}
\frac{\partial \bm{b}^{\rm T}\bm{A}\bm{x}}{\partial \bm{x}} = \bm{A}^{\rm T}\bm{b},\quad
\frac{\partial \bm{x}^{\rm T}\bm{A}\bm{b}}{\partial \bm{x}} = \bm{A}\bm{b},\quad
\frac{\partial \bm{x}^{\rm T}\bm{A}\bm{x}}{\partial \bm{x}} = (\bm{A}+\bm{A}^{\rm T})\bm{x}.\\
\end{align*}
These three rules are actually a special case of a more general rule:

Let the scalar $\alpha$ be defined by $\alpha = \mathbf{u}^{\rm T}\bm{A}\mathbf{v}$~~where\\
 $\mathbf{u} = \mathbf{u}(\bm{x}), \mathbf{v} = \mathbf{v}(\bm{x})$ and $\mathbf{u}: \mathbb{R}^m \rightarrow \mathbb{R}^m, \mathbf{v}: \mathbb{R}^n \rightarrow \mathbb{R}^n$, then
\begin{align*}
\frac{\partial \mathbf{u}^{\rm T}{\mathbf{A}}\mathbf{v}}{\partial \bm{x}} = 
\frac{\partial \mathbf{u}}{\partial \bm{x}}{\mathbf{A}}{\mathbf{v}} + \frac{\partial \mathbf{v}}{\partial \bm{x}}\mathbf{A}^{\rm T}\mathbf {u}.
\end{align*}

Note that there are several conventions in matrix calculus. In this solution, we stick to the \textit{denominator} layout (a.k.a. \textit{Hessian} formulation).
\end{framed}

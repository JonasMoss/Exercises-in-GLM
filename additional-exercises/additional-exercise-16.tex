% Forhåndsvist kildekode for avsnittene 0 til 69


\section*{Additional Exercise 16}

\subsection*{(a)}

The probability mass function of the binomial with parameters $n$
and $\pi$ is

\[
f(v;\pi,n)={v \choose n}\pi^{v}(1-\pi)^{n-v},\:v\in\{0,1,\ldots,n\}.
\]
The log-likelihood, keeping $n$ constant, is 

\[
l(\pi;v)=\log{v \choose n}+v\log\pi+(n-v)\log(1-\pi).
\]
The score equals the derivative of the log-likelihood with respect
to $\pi$, namely

\[
S(\pi;v)=\frac{v}{\pi}-\frac{n-v}{1-\pi}=\frac{v(1-\pi)-\pi(n-v)}{\pi(1-\pi)}=\frac{v-\pi n}{\pi(1-\pi)}.
\]
Since 

\[
\frac{d}{d\pi}\left[\frac{v}{\pi}-\frac{n-v}{1-\pi}\right]=-\frac{v}{\pi^{2}}-\frac{n-v}{(1-\pi)^{2}},
\]
the Fisher information is, using $Ev=n\pi$,

\begin{eqnarray*}
J(\pi) & = & -\frac{n\pi}{\pi^{2}}-\frac{n-\pi n}{(1-\pi)^{2}},\\
 & = & n\left(\frac{1}{\pi}+\frac{1}{1-\pi}\right),\\
 & = & \frac{n}{\pi(1-\pi)}.
\end{eqnarray*}


\subsection*{(b)}

The Wald statistics is 

\[
\frac{\hat{\pi}-\pi}{\sqrt{J^{-1}(\hat{\pi})}}=\sqrt{n}\frac{\hat{\pi}-\pi}{\sqrt{\hat{\pi}(1-\hat{\pi})}}.
\]


\subsection*{(c)}

The score statistic is 

\[
\frac{S(\pi)}{\sqrt{J(\pi)}}=\frac{\frac{v-\pi n}{\pi(1-\pi)}}{\sqrt{\frac{n}{\pi(1-\pi)}}}=\sqrt{n}\frac{\hat{\pi}-\pi}{\sqrt{\pi(1-\pi)}}.
\]


\subsection*{(d)}

The likelihood statistic is

\begin{eqnarray*}
-2\log\Delta & = & -2(l(\pi;v)-l(\hat{\pi};v))=2v\log\left(\frac{\hat{\pi}}{\pi}\right)+2(n-v)\log\left(\frac{1-\hat{\pi}}{1-\pi}\right).
\end{eqnarray*}
When $n$ is large, $-2\log\Delta$ is approximately $\chi_{1}^{2}$-distributed
by Wilks theorem.

\subsection*{(e)}

The solutions are as follows.

\textbf{Wald:}

\begin{eqnarray*}
\sqrt{n}\frac{\hat{\pi}-\pi}{\sqrt{\hat{\pi}(1-\hat{\pi})}} & \approx & -4.4
\end{eqnarray*}

\textbf{Score:}

\begin{eqnarray*}
\sqrt{n}\frac{\hat{\pi}-\pi}{\sqrt{\hat{\pi}(1-\hat{\pi})}} & \approx & -4
\end{eqnarray*}

\textbf{Likelihood:}

\begin{eqnarray*}
-2\log\Delta & \approx & 16.5
\end{eqnarray*}

All tests agree to roughly the same degree that $H_{0}$ is false.
Notice that $\sqrt{16.5}\approx4.05$, and since the root of a $\chi_{1}^{2}$
is the abosolute value of the normal, the likelihood ratio test agrees
with the others as well.

\subsection*{(f)}

The solutions are as follows.

\textbf{Wald:}

\begin{eqnarray*}
\sqrt{n}\frac{\hat{\pi}-\pi}{\sqrt{\hat{\pi}(1-\hat{\pi})}} & \approx & -11.5
\end{eqnarray*}

\textbf{Score:}

\begin{eqnarray*}
\sqrt{n}\frac{\hat{\pi}-\pi}{\sqrt{\hat{\pi}(1-\hat{\pi})}} & \approx & -6.7
\end{eqnarray*}

\textbf{Likelihood:}

\begin{eqnarray*}
-2\log\Delta & \approx & 34
\end{eqnarray*}

Since $\sqrt{34}\approx5.85$, the likelihood test is more conservative.

\subsection*{(g)}

This is the case since 
\[
\sqrt{n}\frac{\hat{\pi}-\pi}{\sqrt{\hat{\pi}(1-\hat{\pi})}}\approx N(0,1)
\]
and $\pm 1.96$ are the two-sided confidence limits for a standard
normal variable.

\subsection*{(h)}

The score is asymptotically standard normal, which implies the square
is asymptotically $\chi_{1}^{2}$. The $(1-\alpha)$-percentile is
$-\Phi^{-1}(\alpha/2)=c_{\alpha}$. We find the confidence interval
by inverting the hypothesis test that rejects when the square of the
score is larger than $c_{\alpha}$. This is done by solving the equaton

\[
n\frac{(\hat{\pi}-\pi)^{2}}{\pi(1-\pi)}=c_{\alpha}^{2}.
\]
The equation is equivalent to

\[
n(\hat{\pi}^{2}-2\hat{\pi}\pi+\pi^{2})=c_{\alpha}^{2}\pi-c_{\alpha}^{2}\pi^{2}.
\]
Rearrange to get

\[
\pi^{2}(n+c_{\alpha}^{2})-\pi(2v+c_{\alpha}^{2})+n\hat{\pi}^{2}=0.
\]
The solutions are

\[
\frac{2v+c_{\alpha}^{2}\pm\sqrt{(2v+c_{\alpha}^{2})^{2}-4(n+c_{a}^{2})n\hat{\pi}^{2}}}{2(n+c_{\alpha}^{2})}.
\]
Simplify to get

\[
\frac{v+\frac{1}{2}c_{\alpha}^{2}}{n+c_{\alpha}^{2}}+\frac{\sqrt{(2v+c_{\alpha}^{2})/4-(n+c_{\alpha}^{2})n\hat{\pi}^{2}}}{n+c_{\alpha}^{2}}.
\]


\subsection*{(i)}
\begin{lyxcode}
>~v~=~30

>~n~=~100

>~alpha~=~0.05

>~CI(n,~v,~alpha)

\$wald

{[}1{]}~0.2101832~0.3898168

\$score

{[}1{]}~0.2189489~0.3958485

\$lrt

{[}1{]}~0.2160263~0.39409
\end{lyxcode}

\subsection*{(j)}
\begin{lyxcode}
>~v~=~5~

>~n~=~100~

>~alpha~=~0.05~

>~CI(n,~v,~alpha)~

\$wald~

{[}1{]}~0.007283575~0.092716425~

\$score~{[}1{]}~0.02154368~0.11175047~

\$lrt~{[}1{]}~0.01823454~0.10438783
\end{lyxcode}

\section*{Additional Exercise 22}

\begin{lstlisting}
> # Enter the data
> lung.cancer.data = data.frame(
+   city = rep(1:4, each = 5),
+   age = rep(1:5, times = 4),
+   cases = c(11,11,11,10,11,13,6,15,10,12,4,8,7,11,9,5,7,10,14,8),
+   number = c(3059,800,710,581,509,2879,1083,923,834,634,3142,
+          1050,895,702,535,2520,878,839,631,539)
+   )
> lung.cancer.data[,"age"] = as.factor(lung.cancer.data[,"age"])
> lung.cancer.data[,"city"] = as.factor(lung.cancer.data[,"city"])
> head(lung.cancer.data)
  city age cases number
1    1   1    11   3059
2    1   2    11    800
3    1   3    11    710
4    1   4    10    581
5    1   5    11    509
6    2   1    13   2879
\end{lstlisting}


\subsection*{(a)}
The expected value of the response variable (\texttt{cases}) is proportional to the total number of male inhabitants (\texttt{number}). We here model the rate, i.e. $\frac{y_{i,j}}{n_{i,j}}$, where $y_{i,j}$ and $n_{i,j}$ are the number of cases and the number of inhabitants in city $i$ and age group $j$. The model $\E\left[\frac{Y_{i,j}}{n_{i,j}}\right] = e^{\eta_{i,j}}$ for a linear predictor $\eta_{i,j}$ hence corresponds to $\E\left[Y_{i,j}\right] = e^{\log n_{i,j} + \eta_{i,j}}$. So, we have to add $\log n_{i,j}$ as an offset to the linear predictor.


\vspace{\baselineskip}
\subsection*{(b)}

\begin{lstlisting}
> Poisson.model = glm(cases ~ offset(log(number)) + age + city, family = poisson, data = lung.cancer.data)
> summary(Poisson.model)

Call:
glm(formula = cases ~ offset(log(number)) + age + city, family = poisson, 
    data = lung.cancer.data)

Deviance Residuals: 
     Min        1Q    Median        3Q       Max  
-1.20728  -0.59302  -0.09784   0.58493   1.46574  

Coefficients:
            Estimate Std. Error z value Pr(>|z|)    
(Intercept)  -5.6455     0.2049 -27.555  < 2e-16 ***
age2          1.0961     0.2483   4.414 1.02e-05 ***
age3          1.5138     0.2317   6.534 6.39e-11 ***
age4          1.7584     0.2295   7.662 1.83e-14 ***
age5          1.8486     0.2354   7.855 4.01e-15 ***
city2        -0.1907     0.1910  -0.999   0.3180    
city3        -0.4791     0.2103  -2.279   0.0227 *  
city4        -0.2534     0.2033  -1.247   0.2125    
---

(Dispersion parameter for poisson family taken to be 1)

    Null deviance: 115.434  on 19  degrees of freedom
Residual deviance:  11.598  on 12  degrees of freedom
AIC: 109.07

Number of Fisher Scoring iterations: 4
\end{lstlisting}
The interpretation of $\widehat{\beta}$ can be done in the same way as in exercise 7.31 a) from the book.

\vspace{\baselineskip}
\subsection*{(c)}
\begin{lstlisting}
> lung.cancer.data[,"Fredericia"] = as.factor(as.numeric(lung.cancer.data[,"city"] == 1))
> head(lung.cancer.data)
  city age cases number Fredericia
1    1   1    11   3059          1
2    1   2    11    800          1
3    1   3    11    710          1
4    1   4    10    581          1
5    1   5    11    509          1
6    2   1    13   2879          0
> # Fit Poisson GLM
> Poisson.model.2 = glm(cases ~ offset(log(number)) + age + Fredericia, family = poisson, data = lung.cancer.data)
> summary(Poisson.model.2)

Call:
glm(formula = cases ~ offset(log(number)) + age + Fredericia, 
    family = poisson, data = lung.cancer.data)

Deviance Residuals: 
     Min        1Q    Median        3Q       Max  
-1.62564  -0.59506  -0.03471   0.17297   1.81669  

Coefficients:
            Estimate Std. Error z value Pr(>|z|)    
(Intercept)  -5.9502     0.1818 -32.736  < 2e-16 ***
age2          1.0997     0.2483   4.429 9.47e-06 ***
age3          1.5187     0.2316   6.556 5.51e-11 ***
age4          1.7671     0.2294   7.704 1.32e-14 ***
age5          1.8582     0.2352   7.899 2.82e-15 ***
Fredericia1   0.2991     0.1606   1.863   0.0624 .  
---

(Dispersion parameter for poisson family taken to be 1)

    Null deviance: 115.434  on 19  degrees of freedom
Residual deviance:  13.663  on 14  degrees of freedom
AIC: 107.14

Number of Fisher Scoring iterations: 4

> 
> # Likelihood ratio test.
> anova(Poisson.model.2, Poisson.model.1)
Analysis of Deviance Table

Model 1: cases ~ offset(log(number)) + age + Fredericia
Model 2: cases ~ offset(log(number)) + age + city
  Resid. Df Resid. Dev Df Deviance
1        14     13.663            
2        12     11.598  2   2.0658
> 1 - pchisq(anova(Poisson.model.2, Poisson.model.1)$Deviance[2], df = 1)
[1] 0.1506362
\end{lstlisting}

The two models are nested. So, we use the likelihood ratio test.\\
p = $0.1506 > 0.05$. So, we keep the null hypothesis and conclude that the model with simplified city effect is a better model.


\vspace{\baselineskip}
\subsection*{(d)}
Rate ratio of new lung cancer cases in Fredericia compared to the three other cities:
$\frac{\E\left[Y|x_{\mathrm{Fredericia}} = 1\right]}{\E\left[Y|x_{\mathrm{Fredericia}} = 0\right]} = \exp\left[\beta_{\mathrm{Fredericia}}\right]$.

\begin{lstlisting}
> # Estimated rate ratio
> exp(summary(Poisson.model.2)$coefficients["Fredericia1","Estimate"])
[1] 1.348685
> # 95% confidence interval of rate ratio
> exp(confint.default(Poisson.model.2))["Fredericia1",]
    2.5 %    97.5 % 
0.9845689 1.8474606 
\end{lstlisting}


\vspace{\baselineskip}
\subsection*{(e)}
\begin{lstlisting}
> # Numeric version of variable age
> lung.cancer.data[,"age.numeric"] = rep(
+   c(mean(40,55), mean(55,60), mean(60,65), mean(65,70), mean(70,75)),
+   times = 4
+   )
> # Fit Poisson GLM
> Poisson.model.3 = glm(cases ~ offset(log(number)) + I(age.numeric-40) + Fredericia, family = poisson, data = lung.cancer.data)
> summary(Poisson.model.3)

Call:
glm(formula = cases ~ offset(log(number)) + I(age.numeric - 40) + 
    Fredericia, family = poisson, data = lung.cancer.data)

Deviance Residuals: 
    Min       1Q   Median       3Q      Max  
-1.8683  -0.6997  -0.1695   0.3869   1.6274  

Coefficients:
                     Estimate Std. Error z value Pr(>|z|)    
(Intercept)         -5.857611   0.158061 -37.059   <2e-16 ***
I(age.numeric - 40)  0.064311   0.006945   9.261   <2e-16 ***
Fredericia1          0.290371   0.160441   1.810   0.0703 .  
---

(Dispersion parameter for poisson family taken to be 1)

    Null deviance: 115.434  on 19  degrees of freedom
Residual deviance:  16.141  on 17  degrees of freedom
AIC: 103.61

Number of Fisher Scoring iterations: 4

> 
> # Likelihood ratio test.
> anova(Poisson.model.3, Poisson.model.2)
Analysis of Deviance Table

Model 1: cases ~ offset(log(number)) + I(age.numeric - 40) + Fredericia
Model 2: cases ~ offset(log(number)) + age + Fredericia
  Resid. Df Resid. Dev Df Deviance
1        17     16.141            
2        14     13.663  3   2.4776
> 1 - pchisq(anova(Poisson.model.3, Poisson.model.2)$Deviance[2], df = 1)
[1] 0.1154789
\end{lstlisting}

The two models are nested. So, we use the likelihood ratio test.\\
p = $0.1155 > 0.05$. So, we keep the null hypothesis and conclude that the model with numerified age effect is a better model.

The interpretation of $\widehat{\beta}_{\mathrm{age~numeric}}$: log rate ratio of one unit increase in \texttt{age numeric}.
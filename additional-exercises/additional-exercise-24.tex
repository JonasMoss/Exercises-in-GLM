\section*{Additional Exercise 24}
\subsection*{(a)}
We already did this many times in earlier exercises.

\vspace{\baselineskip}
\subsection*{(b)}
\subsubsection*{(i)}
$\exp\left[\beta_{\mathrm{\texttt{badh}}}\right] = \frac{\E\left[Y|x_{\mathrm{\texttt{badh}}} = 1\right]}{\E\left[Y|x_{\mathrm{\texttt{badh}}} = 0\right]}$.\\

So, the estimate of rate ratio is
\begin{align*}
\exp\left[\widehat{\beta}_{\mathrm{\texttt{badh}}}\right] = \exp\left[1.1409\right] = 3.1296.
\end{align*}

$95\%$ confidence interval for this rate ratio is
\begin{align*}
&\left[\exp\left[\widehat{\beta}_{\mathrm{\texttt{badh}}} - z_{0.975} \cdot \mathrm{SE}(\widehat{\beta}_{\mathrm{\texttt{badh}}})\right],~\exp\left[\widehat{\beta}_{\mathrm{\texttt{badh}}} + z_{0.975} \cdot \mathrm{SE}(\widehat{\beta}_{\mathrm{\texttt{badh}}})\right] \right]\\
&= \left[\exp[1.1409 - 1.96 \cdot 0.0399], \exp[1.1409 + 1.96 \cdot 0.0399]\right]\\
&= \left[\exp[1.0628], \exp[1.2190]\right]\\
&= [2.8944, 3.3839]
\end{align*}

\subsubsection*{(ii)}
$\exp\left[10\beta_{\mathrm{\texttt{age}}}\right] = \frac{\E\left[Y|x_{\mathrm{\texttt{age}}} = 50\right]}{\E\left[Y|x_{\mathrm{\texttt{age}}} = 40\right]}$.\\

So, the estimate of rate ratio is
\begin{align*}
\exp\left[10\cdot\widehat{\beta}_{\mathrm{\texttt{age}}}\right] = \exp\left[0.0556\right] = 1.0571.
\end{align*}

$95\%$ confidence interval for this rate ratio is
\begin{align*}
&\left[\exp\left[10\cdot\widehat{\beta}_{\mathrm{\texttt{age}}} - 10\cdot z_{0.975} \cdot \mathrm{SE}(\widehat{\beta}_{\mathrm{\texttt{age}}})\right],~\exp\left[10\cdot\widehat{\beta}_{\mathrm{\texttt{age}}} + 10\cdot z_{0.975} \cdot \mathrm{SE}(\widehat{\beta}_{\mathrm{\texttt{age}}})\right] \right]\\
&= [\exp[0.0556 - 1.96 \cdot 0.0168], \exp[0.0556 + 1.96 \cdot 0.0168]]\\
&= \left[\exp[0.0227], \exp[0.0884]\right]\\
&= [1.0230, 1.0924]
\end{align*}

\subsubsection*{(iii)}
$\{\widehat{\mu}|\texttt{age} = 40, \texttt{badh} = 0\} = \exp\left[\widehat{\beta}_{0} + \widehat{\beta}_{\texttt{age}}\cdot40 + \widehat{\beta}_{\texttt{badh}}\cdot0\right] = \exp\left[0.5888 + 0.0056*40\right] = \exp\left[0.810971\right] = 2.2501$

The confidence interval of this rate ratio is equal to the confidence interval of $e^{\eta}$ where $\eta = \beta_{0} + \beta_{\texttt{age}}\cdot40$. We would then first find a confidence interval of $\eta$. This takes the form $\widehat{\eta} \pm z_{1-\frac{\alpha}{2}}\cdot \mathrm{SE}(\widehat{\eta})$, where $\mathrm{SE}(\widehat{\eta}) = \sqrt{\widehat{\Var}(\widehat{\beta}_{0}) + 40^{2}\widehat{\Var}(\widehat{\beta}_{\texttt{age}}) + 2\cdot40\cdot \widehat{\Cov}\left(\widehat{\beta}_{0}, \widehat{\beta}_{\texttt{age}}\right)}$. So, we would need $\Cov\left(\widehat{\beta}_{0}, \widehat{\beta}_{\texttt{age}}\right)$.


\vspace{\baselineskip}
\subsection*{(c)}

\subsubsection*{(i)}
We estimate parameters by solving quasi-likelihood equations instead of the likelihood equations. However, since $\phi$ will cancel, the estimates are the same as for the Poisson model.

\subsubsection*{(ii)}
We compute the Pearson statistic $X^{2} = \sum_{i=1}^{n} \frac{(y_{i} - \widehat{\mu}_{i})^{2}}{\widehat{\mu}_{i}}$ for the Poisson model, and estimate $\phi$ by $\widehat{\phi} = \frac{X^{2}}{n-p}$ where $p = 3$.


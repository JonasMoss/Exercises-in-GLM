\section*{Exercise 1.7}
Recall the definition of \textit{model space}: 
\begin{align*}
C(\bm{X}) = \{\bm{\eta}: \mbox{there is a $\bm{\beta}$ such that $\bm{\eta} = \bm{X}\bm{\beta}$}\}.
\end{align*}
No matter how our $\bm{X}$ looks like, we can always obtain $\bm{\eta} = \bm{X}\bm{\beta}$ by letting $\bm{\beta} = \bm{0}$. So, $\bm{0}$ is in the model space $C(\bm{X})$ for any linear model $\bm{\mu} = \bm{X}\bm{\beta}.$

\textbf{Note}: \textit{Model space} is non-standard terminology, and frequently refers to something else, namely a space of models (such as the space of gamma, exponential and Weibull models). The terminology \textit{column space} is standard, and known from linear algebra.
\section*{Exercise 2.14}
\subsection*{(i)}
In linear model, projection matrix is hat matrix. So,
\begin{align*}
\bm{P}_{X}\bm{X} = \bm{X}(\bm{X}^{\rm T}\bm{X})^{-1}\bm{X}^{\rm T}\bm{X} = \bm{X}.
\end{align*}

\subsection*{(ii)}
By definition of model space, we have
\begin{align*}
C(\bm{X}) = \{\bm{\eta}: \mbox{there exists $\bm{\beta} \in \mathbb{R}^{p}$ such that $\bm{\eta} = \bm{X}\bm{\beta}$}\}
\end{align*}
and
\begin{align*}
C(\bm{P}_{X}) = \{\bm{\tau}: \mbox{there exists $\bm{a} \in \mathbb{R}^{n}$ such that $\bm{\tau} = \bm{P}_{X}\bm{a}$}\}.
\end{align*}

Take an arbitrary $\bm{\tau} \in C(\bm{P}_{X})$. Then, $\bm{\tau} = \bm{P}_{X}\bm{a}$ for an $\bm{a} \in \mathbb{R}^{n}$.
Note that
\begin{align*}
\bm{\tau} = \bm{P}_{X}\bm{a} =  \bm{X}\underbrace{(\bm{X}^{\rm T}\bm{X})^{-1}\bm{X}^{\rm T}\bm{a}}_{\bm{c}} = \bm{X}\bm{c} \in \mathbb{R}^{p} \quad \mbox{and} \quad \bm{c} = (\bm{X}^{\rm T}\bm{X})^{-1}\bm{X}^{\rm T}\bm{a} \in \mathbb{R}^{p}.
\end{align*}
Hence, $\bm{\tau} \in C(\bm{X})$.

The other way follows from $P_X X \beta = X \beta$. Then we have $C(\bm{X}) = C(\bm{P}_{X})$.
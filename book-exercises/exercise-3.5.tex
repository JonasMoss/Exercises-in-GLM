\section*{Exercise 3.5}

Let $\mu$ be the true mean. Then $Y_{i}-\mu_{0}\sim N(\mu-\mu_{0},\sigma^{2})$,
so we can without loss of generality assume that $\mu_{0}=0$ by working
with the proxy variable $\nu=\mu-\mu_{0}$ instead of $\mu$.

Define $P=n^{-1}1_{n}1_{n}^{T}$ and observe that
\begin{eqnarray*}
\frac{1}{\sigma^{2}}Y^{T}PY & \sim & \chi_{1,\sigma^{-2}\mu^{T}P\mu}^{2},\\
\frac{1}{\sigma^{2}}Y^{T}(I-P)Y & \sim & \chi_{n-1}^{2},
\end{eqnarray*}
independently by Cochran's theorem, as the rank of $I-P$ is $n-1$
and the rank of $P$ is $1$. The non-centrality parameter of $\sigma^{-2}Y^{T}(I-P)Y$
is $\sigma^{-2}\mu^{T}(I-P)\mu=0$, as
\begin{eqnarray*}
\mu^{T}(I-P)\mu & = & \mu^{T}\mu-\mu^{T}P\mu,\\
 & = & n\mu^{2}-n^{-1}\mu^{T}1_{n}1_{n}^{T}\mu,\\
 & = & n\mu^{2}-n^{2}\mu,\\
 & = & 0.
\end{eqnarray*}
Regarding the non-centrality parameter of $\sigma^{-2}Y^{T}PY$, define
$\theta=\mu/\sigma$, the \emph{effect size, }used in power analysis
and meta-analysis. Typically, the non-central distributions are functions
of $n\theta^{2}$.

Now observe that $\sigma^{-2}\mu^{T}P\mu=n\theta^{2}$, as 
\begin{eqnarray*}
\sigma^{-2}\mu^{T}P\mu & = & \sigma^{-2}\mu^{T}n^{-1}1_{n}1_{n}^{T}\mu,\\
 & = & n^{-1}\sigma^{-2}(n\mu)^{2},\\
 & = & n\theta^{2}.
\end{eqnarray*}
Then 
\[
\frac{Y^{T}PY}{Y^{T}(I-P)Y}\sim F_{1,n-1,n\theta^{2}},
\]
by the definition of the $F$ statistic. 

Under the null $\mu=0$, the effect size $\theta=0$ the non-centrality
parameter is $0$ as well, so that $F_{1,n-1,n\theta^{2}}=F_{1,n-1}$. 

The \emph{t}-test is arrived at similarily. Since
\[
n^{-1/2}\sigma^{-1}\sum_{i=1}^{n}Y_{i}\sim N(n^{1/2}\theta,1),
\]
and, letting $S$ be the empirical variance,
\[
\sigma^{-2}Y^{T}(I-P)Y=\sigma^{-2}(n-1)S^{2}\sim\chi_{n-1}^{2},
\]
the \emph{t}-statistic
\[
t=n^{-1/2}\sum_{i=1}^{n}Y_{i}/S\sim t_{n-1,\sqrt{n}\theta}.
\]



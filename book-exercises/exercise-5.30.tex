\section*{Exercise 5.30}

A logistic regression model is

\[
E(y_{i})=\textrm{logit}^{-1}(\beta_{0}+\beta_{1}x_{i}),
\]
where $x_{i}=1$ if the $i$th person is a smoker and $0$ if not,
$y_{i}=1$ if the $i$th person has lung cancer, $0$ if not. You
cannot use this model to draw causal conclusions.

The purpose of using controls is to estimatet the causal effect of
smoking. Using controls as in this exercise is called matching or
matched case-control studies.

In matching, we typically want to estimate the average causal effect
of $X$ on $Y$:

\[
E[Y_{X=1}-Y_{X=0}],
\]
where $Y_{X=1}$ is a \emph{counterfactual}, the outcome that would
have happened to patient $Y$ if he had, in fact, smoked. When using
matching, we pretend that the matched person are the same except for
the single factor we want to study, which is smoking in this case. 

The average causal effect can thus be estimated as

\[
\frac{1}{n}\sum(y_{i}[x=1]-y_{i}[x=0]),
\]
where $y_{i}[x=1]$ and $y_{i}[x=0]$ are matched pairs with different
values on the smoking variable.

For more information on causal inference, see e.g. Causal Inference
in Statistics: A Primer (2016) by Pearl et al.